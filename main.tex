\documentclass[]{spie}  %>>> use for US letter paper
%\documentclass[a4paper]{spie}  %>>> use this instead for A4 paper
%\documentclass[nocompress]{spie}  %>>> to avoid compression of citations




%\definecolor{dodgerblue}{rgb}{0.12, 0.56, 1.0}
%\definecolor{coral}{rgb}{1.0, 0.5, 0.31}


\usepackage{amsmath,amsfonts,amssymb}
\usepackage{graphicx}
\usepackage[colorlinks=true, allcolors=blue]{hyperref}
\usepackage[table]{xcolor}
\usepackage{graphicx}
\usepackage{txfonts}
\usepackage{lscape}
\usepackage{hyperref}
\usepackage{comment}
\usepackage{xspace}

\usepackage{bm}
\usepackage{aas_macros}

\renewcommand{\baselinestretch}{1.0} % Change to 1.65 for double spacing

% vettori e versori più carini
\renewcommand\vec{\bm}
\newcommand{\uv}[1]{\vec{\hat{#1}}}

%\newcommand{\degree}{$^{\circ}$\xspace}
%\newcommand{\Fermi}{\textit{Fermi}\xspace}
%\newcommand{\gbm}{\textit{GBM}\xspace}
%\newcommand{\her}{\textit{HERMES}\xspace}
%\newcommand{\us}{$\mu s$\xspace}
%\newcommand{\mq}{$m^{2}$\xspace}


\def \her{\textit{HERMES}\xspace}
\def \Fermi{\textit{Fermi}\xspace}
\def \gbm{\textit{GBM}\xspace}
\def \us{$\mu s$\xspace}
\def \mq{$m^{2}$\xspace}
\def \degree{$^{\circ}$\xspace}

\title{TIMING TECHNIQUES APPLIED TO DISTRIBUTED MODULAR HIGH-ENERGY ASTRONOMY: THE HERMES PROJECT}

\author[a]{A.~F.~Gambino}
\author[b]{A.~Sanna}
\author[a]{L.~Burderi}
\author[a]{A.~Riggio}
\author[b]{T.~Di Salvo}
\author[c]{F.~Fiore}
\author[d]{M.~Lavagna}
\author[e]{R.~Bertacin}
\author[f]{Y.~Evangelista}
\author[g]{R.~Campana}
\author[g]{F.~Fuschino}
\author[d]{P.~Lunghi}
\author[h]{A.~Monge}
\author[e]{B.~Negri}
\author[e]{S.~Pirrotta}
\author[e]{S.~Puccetti}
\author[i]{F.~Amarilli}
\author[f]{F.~Ambrosino}
\author[j,k]{G.~Amelino-Camelia}
\author[a]{A.~Anitra}
\author[a]{M.~Barbera}
\author[d]{M.~Bechini}
\author[l]{P.~Bellutti}
\author[m]{G.~Bertuccio}
\author[m]{J.~Cao}
\author[f]{F.~Ceraudo}
\author[n]{T.~Chen}
\author[o]{M.~Cinelli}
\author[p]{M.~Citossi}
\author[q]{A.~Clerici}
\author[d]{A.~Colagrossi}
\author[d]{S.~Cruzel}
\author[p]{G.~Della Casa}
\author[l]{E.~Demenev}
\author[r]{M.~Del Santo}
\author[p]{G.~Dilillo}
\author[s]{P.~Efremov}
\author[f]{M.~Feroci}
\author[c]{C.~Feruglio}
\author[m]{F.~Ferrandi}
\author[t]{M.~Fiorini}
\author[d]{M.~Fiorito}
\author[s]{D.~Gacnik}
\author[u]{G.~Galg\'oczi}
\author[n]{N.~Gao}

 
%M. Gandolaj, G. Ghirlandaa, A. Gomboco, M. Grassir, A. Guzmans, M. Karlicao, 
%U. Kosticn, C. Labantia, G. La Rosaa, U. Lo Ciceroa, B. Lopez Fernandeze, P. Malcovatir, 
%A. Masellid, A. Mancab, F. Melej, D. Milánkovicht, G. Morgantea, L. Navaa, P. Nogaraa, M. Ohnoq, D. Ottolinac, A. Pasqualec, A. Palu, M. Perria,d, M. Piccininc, R. Piazzollaa, S. Pliego-Caballeroq, 
%J. Prinettoc, G. Pucaccol,  A. Rachevskiu, I. Rashevskayau,v, A. Riggiob, J. Ripaq, F. Russoa, 
%A. Papittoa, S. Piranomontea, A. Santangelos, F. Scalac, D. Selcanp, S. Silvestrinic, G. Sottilea, 
%T. Rotovnikp, C. Tenzers, I. Troisic, A. Vacchik, E. Virgillia, N. Wernerx,q, G. Zampau, N. Zampav,m, G. Zanottic, L. Wangk, Y. Xuk

\affil[a]{Dipartimento di Fisica, Universit\`a degli Studi di Cagliari, SP Monserrato-Sestu km 0.7, I-09042 Monserrato, Italy}
\affil[b]{Dipartimento di Fisica e Chimica, Universit\`a degli Studi di Palermo, via Archirafi 36, I-90123 Palermo, Italy}
\affil[c]{INAF-OATS, Via G.B. Tiepolo 11, I-34143, Trieste, Italy}
\affil[d]{Politecnico di Milano, Department of Electronics, Information and Bioengineering, Via Anzani 42, I-22100 Como, Italy;}
\affil[e]{Agenzia Spaziale Italiana, via del Politecnico snc, 00133 Roma, Italy}
\affil[f]{INAF-IAPS Rome, Via del Fosso del Cavaliere 100, I-00133, Italy}
\affil[g]{INAF-OAS Bologna, Via Gobetti 101, I-40129, Bologna, Italy}
\affil[h]{DEIMOS, Spain}
\affil[i]{Fondazione Politecnico di Milano, Piazza Leonardo da Vinci, 32 20133 Milano, Italy}
\affil[j]{Dipartimento di Fisica Ettore Pancini, Universit`a di Napoli ”Federico II”}
\affil[k]{INFN, Sezione di Napoli, Complesso Univ. Monte S. Angelo, I-80126 Napoli, Italy}
\affil[l]{Fondazione Bruno Kessler – FBK, Via Sommarive 18, I-38123 Trento, Italy}
\affil[m]{Department of Electronics, Information and Bioengineering (DEIB) of Politecnico di Milano, Como Campus, Via Anzani 42, 22100 Como, Italy}
\affil[n]{Institute of High Energy Physics, Chinese Academy of Sciences}
\affil[o]{Dipartimento di Matematica, Università di Roma Tor Vergata}
\affil[p]{Universit\`a degli Studi di Udine, Via delle Scienze, 206, 33100 Udine, Italy}
\affil[q]{Aalta Lab, Slovenia}
\affil[r]{INAF, Istituto di Astrofisica Spaziale e Fisica Cosmica, Via U. La Malfa 153, I-90146 Palermo, Italy}
\affil[s]{University of Nova Gorica, Slovenia}
\affil[t]{INAF-IASF Milano, Via Bassini 15, I-20100 Milano, Italy}
\affil[u]{Skylabs, Slovenia}


\authorinfo{Further author information: (Send correspondence to A.A.A.)\\A.A.A.: E-mail: aaa@tbk2.edu, Telephone: 1 505 123 1234\\  B.B.A.: E-mail: bba@cmp.com, Telephone: +33 (0)1 98 76 54 32}

% Option to view page numbers
%\pagestyle{empty} % change to \pagestyle{plain} for page numbers   
\pagestyle{plain}
\setcounter{page}{1} % Set start page numbering at e.g. 301
 
\begin{document} 
\maketitle

\begin{abstract}
The HERMES-TP/SP (High Energy Rapid Modular Ensemble of Satellites Technologic and Scientific Pathfinder) is an in-orbit demostration of the so-called \emph{distributed astronomy} concept. Conceived as a mini-constellation of six 3U nano-satellites hosting a new miniaturized detector, HERMES-TP/SP aims at the detection and accurate localisation of bright high-energy transients such as Gamma-Ray Bursts. The large energy band, the excellent temporal resolution and the wide field of view that characterised the detectors of the constellation represent the key features for the next generation high-energy all-sky monitor with good localisation capabilities that will play a pivotal role in the next decade on multi-messenger astronomy. 
In this work, we will describe in detail the temporal techniques that allow the localisation of bright transient events taking advantage of their almost simultaneous observation by spatially spaced detectors. Moreover, we will quantitively discussed the all-sky monitor capabilities the HERMES Pathfinder as well as its achievable accuracies on the localisation of the detected GRBs. 
\end{abstract}


\keywords{Gamma Ray Bursts, X-rays, CubeSats, nano-satellites, temporal triangulation}

\section{INTRODUCTION}


The HERMES (High Energy Rapid Modular Ensemble of Satellites) project is based on a revolutionary mission concept based on modular astronomy. More specifically, HERMES is conceived as a constellation of nano/micro/small satellites in low Earth orbits, hosting technologically advanced X/gamma-ray detectors, with relatively small effective area, to localise and investigate the spectral and temporal properties the high-energy emission of bright highly-energetic transient events.

Interestingly, we are now in the middle of a transition phase both scientifically and technologically speaking. On one hand, we recently witness the beginning of the so-called \emph{Multi-messenger Astronomy}, that coincided with the observation of the Gravitational Wave Event GW170817 by the Advanced LIGO/Virgo [\citenum{GW170817_GW}], followed by the detection of the associated short Gamma-Ray Burst (GRB) detected by the Fermi and INTEGRAL satellites [\citenum{Troja17}], as well as the countless follow-up multi-wavelength observations. Immediately after this discovery, it become even clearer for the community the urgency for an high-energy all-sky with good localisation capabilities to work in parallel with gravitational wave observatories such as Ligo/Virgo/Kagra that will reach their final sensitivity within a few years. At the moment, all the available satellites providing the detection and localisation of GRBs have been operational for more than a decade with large chances of decommissioning in the next years, while next-generation large-area detectors will not be available for at least 10-15 years from now. 

On the other hand, we are living in the hera where new technologies can finally challenge the current assumptions that high-energy astronomical observations from space can only be achieved by big satellites, usually designed, built, launched and managed by government space agencies. CubeSats, light spacecrafts with small sizes and reduced costs, are now considered a competitive solution for space applications as they allow equilibrium among crucial variables of a space project, such as development time, cost, reliability, mission lifetime, and replacement. Interestingly, they represent the key to realise the revolutionary concept of distributed (modular) space astronomy crucial to investigate the unknown of the Universe. Indeed, the large number of photons required to performe cutting edge (astro)physical space science can be collected adding up the contribution of a large number of astrophysical detectors distributed over a fleet of nano/micro/small-satellites allowing to achieve huge overall collecting areas which otherwise would be unreachable with a single instrument and impossible to carry with the current rocket load capabilities. The decreasing trend of producing costs as well as the increase of launching opportunities witnessed in recent years, allows us to concretely conceive fleets of hundreds/thousands of coordinated satellites acting as a single detector of unprecedented collecting power.


The HERMES Full Constellation (HFC) has been conceived to tackle three main scientific objectives: 
\begin{itemize}
\item the accurate and prompt localization of bright hard X-ray/soft gamma-ray transients such as GRBs. Fast high-energy transients are among the likely electromagnetic counterparts of the gravitational wave events (GWE) recently discovered by Advanced LIGO/Virgo, and of the Fast Radio Burst; 
\item Open a new window of timing at X/gamma-ray energies, and thus investigate for the first time the temporal structure of GRBs down to fractions of micro-seconds, to constrain models for the GRB engine;
\item Test quantum space-time scenarios by measuring the delay time between GRB photons of different energy.
\end{itemize}

The determination of the position in the sky of an astrophysical transient source is crucial to investigate its origin. A clear example of that is the case of Gamma Ray Bursts (GRBs, hereinafter). Indeed, for almost thirty years after their discovery, GRBs remained poorly understood due to the lack of precise localization. Only in the late 90s, with the launch of the Italian-Dutch X-ray satellite BeppoSAX (ref), the detection of the GRB X-ray afterglow (van Paradijs, J., et al., 1997 Nature, 386, 686) and the subsequent identification of the optical and radio transient by ground-based telescopes, allowed the identification of the host galaxy and hence its extragalactic origin (see e.g. [\citenum{Schilling02}], for a review). This paramount discovery allowed to gauge the energy output of the GRBs, establishing their cosmological nature. GRBs proved to be, eventually, the most powerful electromagnetic explosions in the Universe, one of the best tools to investigate the Universe during its infancy. Similarly, the recent confirmation of short GRBs being the electromagnetic counterparts of a Gravitational Wave Events (see e.g. [\citenum{GW170817_GW,Troja17}]) made the improvement of localisation capabilities of the astronomical observatories a short term technological priority. Indeed, the key to exploiting the “Multi-messenger Astronomy” is the fast and accurate identification of the electromagnetic counterparts of GWEs, that is crucial to promptly characterised the properties of these events. 

The improved sensibilities of the next generation interferometers, will allow the detection of fainter and further away events, making more challenging the identification of their electromagnetic counterparts given the larger portions of space to be searched. Taking as a reference the famous GW170817 event, the horizon for NS-NS merging events detected with a similar signal to noise will reach up to 200 Mpc for LIGO and 100-130 Mpc for Virgo in a few years from now. This implies an increase of the discovery volume a factor $\sim$100 with respect to the GW170817 case. In this scenario, the operation of an efficient X/gamma-ray all-sky-monitor with good localization capability will have a pivotal role in quickly discovering the high-energy counterparts of GWEs and triggering coordinate multi-wavelength observing campaigns.

Another key aspect of modular astronomy is the possibility to combine together the information collected by single elements of the constellation to increase the signal-to-noise ratio, recreating the capabilities of a single observatory with large collecting area. Indeed, once the transients are detected and localised, the signals received by the different detectors can be combined together, after correcting for the delay time of arrivals, significantly increasing their statistics and thus the sensitivity to finer temporal structures. This aspect is particularly important for events such as GRBs,  characterised by huge luminosities and fast variability, often as short as one millisecond. Best most accurate available description of these events is included in the so-called \emph{fireball model}, i.e. a relativistic bulk flow where shocks efficiently accelerate particles. Interestingly, while successful in explaining GRB observations, this model implies a thick photosphere, hampering direct observations of the hidden inner engine that accelerate the bulk flow. One possibility to shed light on their inner engines is through GRB fast variability (see e.g. Kobayashi et al. 1997, ApJ, 490, 92; Morsony et al. 2010, ApJ, 539, 712; Nakar \& Piran 2002, ApJ Letters, 572, L139). GRB light-curves have been investigated in detail down to $sim$1 ms or slightly lower (see e.g. Walker et al. 2000, ApJ, 537, 264; MacLachlan et al. 2013, MNRAS, 432, 857), while the $\mu$s-ms window is basically unexplored, as little known is also the real duration of the prompt event. We do not know how many shells are ejected from the central engine, which is the frequency of ejection and which is its length. With HFC it will be possible to possible to access the the $\mu$s-ms timing window for GRBs, allowing to further investigate their central engines.


The extraordinary capabilities of the HFC will also allow the first dedicated experiment for testing quantum gravity theories. The experiment is based on the predictions of a discrete structure for space on small scales (of the order of the Plack length) proposed by several theories. This space discretization implies the onset of a dispersion relation for photons, as well as an energy dependence of their propagation speed. A promising method for constraining a first order dispersion relation for photons in vacuo is the study of discrepancies in the arrival times of high-energy photons of GRBs emitted at cosmological distances in different energy bands (see e.g. Burderi et al. 2020 and references therein for more details on the topic).



Taking advantage of the modularity of the project, the HERMES concept will be tested following a step-by-step strategy. More specifically, at first it will be realised the HERMES Pathfinder with the aim at proving in space the HERMES concepts, by detecting and localizing GRBs with six units. The successful realization of this experiment will guide the consolidation of the HERMES full constellation design with  the final goal to monitor the full sky and provide <arcmin localization of most GRBs.
The HERMES Pathfinder consist of two different projects: the HERMES Technological Pathfinder (HTP) and the HERMES Scientific Pathfinder (HSP). The former, funded by the Italian Ministry of University and Research (MIUR) and the Italian Space Agency (ASI), aims at producing three 3U nano-satellites equipped with X/gamma-ray detectors [\citenum{Fuschino20,Evangelista20}]. The latter, funded by the European Union’s Horizon 2020 Research and Innovation Program, aims at realising three additional units (payload+spacecraft). Moreover, the project includes the design and development the mission and science operation centers (MOC \& SOC). 
The HTP/HSP mini-constellation of six 3U units should provide enough GRB detections and localizations to:
\begin{itemize}
	\item validate the overall concept as well as study the statistical and systematic uncertainty on both detection and localization to design the full constellation;\\ 
	\item prove that accurate timing in the still little explored window $\mu$s-ms is feasible using detectors with relatively small collecting area;\\
	\item study uncertainties associated to the addition of the signal from different detectors to improve the statistics on high-resolution time series. 
\end{itemize}

Finally, ASI recently approved and funded the participation to the project SpIRIT (Space Industry Responsive Intelligent Thermal), founded by the Australian Space Agency, and led by University of Melbourne. SpIRIT will host an HERMES-like detector and S-band transmission systems. The HERMES-TP/SP mini-constellation of six satellites plus SpIRIT should be tested in orbit during 2022. More details on the overall HERMES mission description can be find in [\citenum{Fiore20}].


\begin{figure}
\begin{center}
\begin{tabular}{c}
\includegraphics[height=8cm]{triangulation} 
\end{tabular}
\end{center}
\caption[example] 
%>>>> use \label inside caption to get Fig. number with \ref{}
{ \label{fig:triangulation} 
Schematic representation of the \emph{Temporal Triangulation} principle applied to a constellation of CubeSats distributed in low Earth orbits. Red arrows represent the emitted GRB photons, while the red dashed lines describe the travelling GRB front wave at different times. The green segment represents the difference in travelled distances of the GRB photons detected by two CubeSats placed at a generic baseline (white dashed line).}
\end{figure} 


\section{Triangulation technique}
The aim of this work is to investigate the localisation capabilities of the HTP/HSP mini-constellation composed of six 3U units. The main targets that will be discussed in the following are the highly energetic high-energy transients GRBs. 

The simple and robust idea that will be applied for accurately localise the transient astrophysical sources is the so-called \emph{Temporal Triangulation}.  

To describe the principle behind the method, let us represent the transient event as a narrow wavefront (pulse) travelling in a given direction and let us displace a network of detectors in space. The narrow wavefront will hit the detectors of the network at different times that depend on the spatial position of each detector and the direction of the wavefront. As represented in Figure~\ref{fig:triangulation}, the transient event will be registered by two detectors of the network with a delay $dt$ proportional to their projected distance with respect to the source direction (green segment). The combination of the delays measured by different pairs of detectors observing the same event will allow to reconstruct its position in the sky. For the sake of description, let us consider a subset of 3 detectors distributed in the equatorial plane (panel a of Figure~\ref{fig:triangulation2}) observing simultaneously a generic event $S$ in certain position in the sky (with direction with respect to the Earth barycenter represented by the yellow dashed line). As shown in panel b of Figure~\ref{fig:triangulation2}, the delay measured $\Delta t_{BC}$ by combining the observations of the detectors B and C allows us to identify a set of infinite possible directions of the transient source belonging to a circulare base of a cone. Each of these source directions $\vec{d}_i$ satisfies the relation $\Delta t_{BC}=\vec{\rho}_{BA}\cdot \vec{d}_i$, where $\vec{\rho}_{BA}$ is the vector describing the distance between the two detectors. A similar result can be obtained by combining the measurements of the detectors B and C'. Interestingly, the superposition of these two results reduces the degeneracy in the source direction, allowing us to determine two possible positions of the source defined as the intersections of the two cones (Figure~\ref{fig:triangulation2} panel c), one of which corresponds (as expected) with the real one. Increasing the number of independent delay measurements, it is then possible to univocally localise the transient event in the sky. It is then clear that in order to achieve localisation capabilities, a generic fleet of the detectors distributed in space should guarantee the simultaneous observation of an event with at least three of its elements.    

\begin{figure}
\begin{center}
\begin{tabular}{ccc}
\includegraphics[height=5cm]{trinag_fig1} &
\includegraphics[height=6cm]{triang_fig2} &
\includegraphics[height=6cm]{triang_fig3} \\
a) & b) & c)\\
\end{tabular}
\end{center}
\caption[example] 
%>>>> use \label inside caption to get Fig. number with \ref{}
{ \label{fig:triangulation2} 
\emph{Panel a)} Schematic representation of a triplet of CubeSats distributed in an equatorial plane that observed simultaneously a transient event located at the position $S$. \emph{Panel b)} Graphic representation of the infinity source directions identified using the delay measured by combining the observations of the event obtained with the detectors B and C. \emph{Panel C)} Superposition of the source directions obtained combining the delays obtained with the detector paris B-C and B-C'. The intersection of the two cones identifies two possibile locations of the event in the sky.}
\end{figure}

The description of the method reported so far does not take into account of any possible source of the uncertainty associated with the system. More in detail, the localisation capabilities of the system, hence the accuracy associated with the source position, will depend on several aspects such as the capability of reconstruct the position of the detectors during the observation of the event, the ability to recover the delay between signals observed by different detectors and the capability of the detector to precisely time tag the photons associated with the transient event. 
A first proxy on the accuracy in determining the source position $\sigma_{PA}$ can be determine in the hypothesis of an event (e.g. a GRB) whose emitted photons arrive to a series of N detectors uniformly distributed in an orbit, and it is given by the expression:

\begin{equation}
\label{eq:sigma_pos}
\sigma_{PA} = \dfrac{\sqrt{(\sigma^2_{delay}+\sigma^2_{tpos}+\sigma^2_{time})}}{<Baseline>\sqrt{(N-1-2)}},
\end{equation}
%
where $\sigma_{delay}$ is the error on the delay measurement obtained combining the light-curves recorded by two detectors, $\sigma_{tpos}=\sigma_{pos}/c$ is the error induced by the uncertainties on the space localisation of the detectors, $\sigma_{time}$ is the uncertainty on the absolute time reconstruction, $<Baseline>$ is the average distance between the detectors ad N$_{ind} = N-1$ is the number of statistically independent pairs of satellites used to determine the delay measurements.\\

	
%%This relation highlights how the positional accuracy, in addition to the number of satellites making up the swarm, is strongly dependent from the uncertainty on the cross-correlation delay measurement E$_{cc}$. In order to obtain a realistic estimation of this uncertainty as a function of the GRB duration, flux and of its intrinsic variability, we have to take advantage of the current repositories of both short and long GRB data, in order to simulate as much as possible the wide phenomenological properties characterizing these transient events, as seen by the future \her mission. 


For a more accurate approach on determine the position and relative uncertainties of a generic GRB in the sky by means of time delay measurements, let us consider a swarm of $n$ satellites, each one identified by a position vector $\vec{r}_i$ (with $i=0,\dots, n-1$) with respect to a suitable reference frame, e.g. the Earth barycenter in equatorial coordinates.
To determine the GRB direction $\hat{d}$, it is possible to measure the time delays of the GRB signal as seen from each pair of  satellites. 
Defining $t_0$ the time at which the GRB signal arrives at the origin of the chosen reference frame, each $i$-th satellite will receive the GRB at a time $t_i$
\begin{equation}
  t_i = t_0 - \frac{\vec{r}_i \cdot \hat{d}}{c}.
\end{equation}
The expected time delays between two satellites will be
\begin{equation}
    \Delta t_{ij}(\hat{d}) \equiv t_j - t_i = \frac{(\vec{r}_j - \vec{r}_i) \cdot \hat{d}}{c} =  \frac{\vec{\rho}_{ij} \cdot \hat{d}}{c},
\label{eq:delayexp}
\end{equation}
where $\vec{\rho}_{ij} \equiv \vec{r}_j - \vec{r}_i$.
The real (measured) time delay between the signal recorded by two satellites $\Delta \tau_{ij}$ is inferred e.g. by applying cross-correlation techniques to the light-curves.
The direction of the GRB, e.g. its equatorial coordinates, can be estimated comparing the computed and measured delays between satellites using e.g. the non linear least squared method.
We define the $\chi^2(\hat{d})$ function as the sum the squares of the difference between the expected and observed time delay divided by its statistical error
  \begin{equation}
    \chi^2(\hat{d}) = \sum_{i = 0}^{n - 2}\sum_{j = i + 1}^{n - 1} \frac{(\Delta \tau_{ij} - \Delta t_{ij}(\hat{d}))^2}{\Theta_{ij}^2},
    \label{eq:chisq}
  \end{equation}
where $\Theta_{ij}^2$ includes the positional error on the satellites expressed in light-seconds, the accuracy in the absolute timing of the detectors, the uncertainty on constraining the time delay between the signals and any hypothetical systematic uncertainty related to the set-up or method applied.
  
The unitary vector $\hat{d}$ identifying the GRB direction can be written in terms of Right Ascension $\alpha$ and Declination $\delta$, that is
  
  \begin{equation}
    \hat{d} = \{\cos{\alpha} \cos{\delta}, \sin{\alpha} \cos{\delta}, \sin{\delta}\}.
  \end{equation}

Minimising Eq. \ref{eq:chisq} with respect to $\alpha$ and $\delta$ gives us an estimate of the direction of the GRB.
Moreover, if Eq. \ref{eq:chisq} satisfies all the hypotheses in [Y.  Anvi, 1976, ApJ, 210, 612], we can also calculate the confidence region for the GRB equatorial coordinates on the plane of the sky.






\section{EXPLOITING THE HERMES PATHFINDER LOCALISATION CAPABILITIES }

In the following, we describe in details the analysis performed, as well as the related assumptions and caveats, to investigate the capabilities of the HERMES-TP/SP mini-constellation (6 3U CubeSats) on localizing GRBs in the sky.

The key to accurately locate an event by means of the temporal triangulation method described above is to decrease as much as possible the uncertainties summarised by the term $\Theta^2$ of Eq.\ref{eq:chisq}. Starting from the HTP/SP technical properties, we can investigate the positional uncertainty budget to be able to identify the most crucial limiting factors. The final design of the spacecraft including GPS receivers and accelerometers, guarantees the possibility to reconstruct the position of the CubeSats with an accuracy smaller than 30 meters, that translates into an temporal accuracy lower than 30 ns (see SPIE PoliMi). Moreover, the absolute timing accuracy achievable from the detector is going to be lower than 0.4 $\mu$s for both the X and S modes [see SPIE Evangelista]. As we show in more detail later, considering the HERMES-TP/SP set up, we can conclude that, even for the brightest GRBs, the uncertainty on the GRB position will be dominated by the accuracy on the time delay between the GRB light-curves and possibly by unknown systematics still to be investigated. In the following we will show that on average uncertainties on the time delays obtained applying cross-correlation techniques are of the other of tenths of milliseconds, a few orders of magnitude larger than the uncertainties discussed above.

\subsection{GRB structure and time delay accuracy}
\label{sec:grbcc}
To be able to investigate the achievable accuracy in the measurement of the time delays between the arrival times for photons emitted by a generic GRB and observed by different detectors of the HTP/SP mini-constellation, we built a procedure that includes the creation of GRB templates, the application of cross-correlation techniques as well as Monte Carlo simulations.

As a first step, we searched the available Fermi GBM archive seeking for GRBs characterised by variability on time scales as short as a few milliseconds. The hypothesis being that fast variability should enhance the sensitivity on time delay measurements, especially when the statistics of the available data is relatively limited. We then isolated two candidates, one belonging the so-called short GRBs and the other from the long class. More specifically, the short GRB (id. GRB120323507) has been observed on 2012 March 23, and it is characterised by a $t_{90}$\footnote{Time interval in which the integrated photon counts increase from 5\% to 95\% of the total counts.} duration of $\sim0.4$ seconds with a fluence of $\sim1\times10^{-5}$ erg cm$^{-2}$. On the other hand, the long GRB (id. GRB13052327) has been detected on 2013 May 2, and it is characterised by a $t_{90}$ duration of $\sim24$ seconds with a fluence of $\sim1\times10^{-2}$ erg cm$^{-2}$.

The data collected from the GBM catalogue includes light-curves from the sodium iodide (NaI) scintillators and from the cylindrical bismuth germanate (BGO) scintillators, both having a collecting area of about 125 cm$^2$. The NaI detectors are sensitive to energies included between few keV up to about 1 MeV, while the BGO detectors cover the energy range 150 keV to 30 MeV. We selected data captured with the so called \textit{Time-tagged event (TTE)} format, where the GRBs are continuously recorded with a time resolution of 2\us, within an time interval that includes 15-30 s of pre-trigger information and about 300 s of data after the trigger time.

To be able to recreate the GRB light-curves as seen by the HTP/SP detectors, we selected the energy range 50-300 keV at which corresponds to the largest effective area of scintillators (around 50 cm$^2$). For this reason, data collected from the GBM observations has been previously filtered in order to have events only in this energy range.

The need to generate GRB templates (functional forms of the GRB light-curves) comes from two crucial aspects in the procedure follow to investigate the measurement of signal delays: a) flexibility to recreate GRB light-curves independently of the detectors effective area and b) the possibility to simulate GRB light-curves with intrinsically poor statistics.  

Indeed, simulations on short time scales ($\sim$1 ms) of a unique-like type of transient events such as a GRB, based on observed light-curves, can be challenging when the effective area of the detector is so small that the statistic is fully dominated by Poissonian fluctuations that unavoidably characterised the (quantum) detection process. In particular, if the detected counts within the given time scale is $\leq1$, quantum fluctuations of the order of 100\% are expected. If, naively, the number of counts per bin is simply rescaled to account for an increase effective area, these quantum fluctuations can introduce a false imprint of 100\% variability with respect to the original signal. No definite cure is available to mitigate this problem, that could be, however, alleviated by rebinning and/or smoothing techniques. Although smoothing techniques allows the creation of light curves for a desired temporal resolution, correlation between subsequent bins is unavoidable. Cross-correlation techniques are strongly biased by this effect, therefore we opted for a more conservative method implying standard rebinning in which the number of photons accumulated in each (variable) bin is fixed. After several trials and Monte-Carlo simulations, we found that 6 photons per bin allows to preserve the signal variability introducing undesired fluctuations not larger than $\sim$30\%. Applying this rebinning techniques to the GBM light-curves (at the maximum time resolution of 2\us), we generated a variable bin size light-curves. In order to generate a template usable on any time scale, we linearly interpolated the previous light-curve to create a functional expression (template) for the theoretical light-curves. We note explicitly, that linear interpolation between subsequent bins is the most conservative approach that does not introduce spurious variability on any time scales. For a given temporal bin size, it is then possible to rescale the GRB template previously described in order to match the requested effective area (e.g. that of the HTP/HSP detectors), generating then the expected number of photons within the time bins. In addition, before and after the burst, we rescaled the background on the GRB template to match the nominal background collected the detector as predicted with respect to the CubeSat orbits. Figure~\ref{fig:template_short_long} shows the templates (red lines) for the long (left panel) and short (right) GRBs generated by following the procedure described above. 


\begin{figure*}[h!]
\centering
\includegraphics[scale=0.52,angle=0]{fig/Template_comparison_Long.pdf}
\includegraphics[scale=0.52,angle=0]{fig/Template_comparison_Short.pdf}

\caption{The Fermi/GBM light curves of the long GRB 130502327 (left panel) and of the short GRB 120323507 (right panel) and the relative \textit{template} (red line) obtained with the procedure described in the text. In both cases, for reasons of clarity, we used a bin time of 10$^{-2}$ s for the light curves and templates of both the GRBs.} 
\label{fig:template_short_long}
\end{figure*}

Starting from the GRB templates we generated light-curves by rescaling the detector effective area to match that of the HTP/HSP mini-constellation and by applying a Poissonian randomization of the counts contained in each bin of the template. We then applied standard cross-correlation techniques (\textbf{ref}) using two light-curves with the aim to determine the time delay between two signals. Since we are interested in reconstructing the accuracy achievable in the time delay, and not stringily on the delay per se, we did not shift in time the template to simulate the signals. To extract the temporal information of the delay, we fitted a restricted region around the peak of the cross-correlation function with an \emph{ad hoc} model composed of an asymmetric double exponential component. The uncertainty associated with the location (in delay) of peak of the cross-correlation function defines the accuracy on determine the delay between the two detected signals. In Figure~\ref{fig:xc_profiles} we report the cross-correlation functions obtained by cross-correlating the simulated light-curves of the long GRB 130502327 (left panel) and the short GRB 120323507 (right panel) as seen by the HTP/HSP detectors assuming an on-axis detection. Moreover, in the insets of Figure~\autoref{fig:xc_profiles} we report the zoom of the peak of the cross-correlation functions and the relative best-fitting model (red solid line).
	


\begin{figure}[h!]
\centering
\includegraphics[scale=0.52,angle=0]{fig/CrossCorrelation_fit_LONG.pdf}
\includegraphics[scale=0.52,angle=0]{fig/CrossCorrelation_fit_Short.pdf}

\caption{Cross-correlation functions obtained by simulating the GRB light-curves of the long GRB 130502327 (left panel) and of the short GRB 120323507 (right panel) using the templates shown in \autoref{fig:template_short_long} rescaled to match the effective area of the HTP/HSP detectors. The insets report a zoom-in of the cross-correlation profiles around the peak as well as their best-fitting model (red solid line).} 
\label{fig:xc_profiles}
\end{figure}

\emph{How reliable is the fitting of the cross-correlation peak in terms of determining the accuracy of time delays between the two light-curves?} Given the complexity of an analytic approach to the problem, we decided to tackle the issue taking advantage of Monte Carlo simulations. More precisely, for each GRB, we generated 2000 light-curves by means of Poissonian randomisation of the template are we generated 1000 cross-correlation functions. For each cross-correlation function, we then determined the delay between the light-curves by fitting the peak of the function as described above. From the overall distributions of delays obtained for the long and short GRBs(left and right panels in \ref{fig:xc_sigma}) we estimated the standard deviations $\sigma_{cc-short}\sim1.2\times10^{-3}$s and $\sigma_{cc-long}\sim 8.2\times10^{-5}$s, respectively, that we interpret as a realistic estimate of the accuracy on the time delay measured with this procedure. It is worth noting that for the long GRB the mean uncertainty (obtained by averaging the results from the 1000 simulations) on the time delay obtained by fitting the peak of the cross-correlation function is only 20\% larger than with respect to  the sigma of the time delay distribution. On the other hand, for the short GRB the uncertainty on cross-correlation peak is almost a factor of 4 smaller with respect to the sigma of the peak distribution \textbf{da verificare}.


\begin{figure}[h!]
\centering
\includegraphics[scale=0.45,angle=0]{sigma_cc_sim}
\vspace{-1cm}
\caption{Distribution of delays obtained applying cross-correlation techniques to pairs of simulated light curves of the long (left panel) and short (right panel) GRBs rescaled to match the HTP/HSP effective area. The distributions are the result of 1000 Monte-Carlo simulations (see text for more details). The overlaid red line represents the best-fit normal distribution to the data.} 
\label{fig:xc_sigma}
\end{figure}

\emph{How does the GRB morphology affect the capability to accurately determine time delays by cross-correlation techniques?}
As a first attempt to test the dependence of the cross-correlation uncertainty on the brightness and temporal structure of the GRBs, we applied the technique described above to two unbiased random samples each including 100 short and 100 long GRBs selected from the available Fermi GBM catalogue.
The randomness of the samples guarantees a good coverage of the vast variety of phenomenologies, fluxes, durations and intrinsic variability that were recorded during the Fermi mission up to the moment in which this paper was written. For the long GRBs, the sample includes bursts having net fluxes ranging between 0.16 and 26 ph cm$^{-2}$ s$^{-1}$, durations between 3 and 138 s, and fluence values between 1.67 and 4 ph cm$^{-2}$. On the other hands, the sample of short GRBs ranges in flux between 0.6 and 188 ph cm$^{-2}$ s$^{-1}$, durations between 0.03 and 1.9 s and fluence values between 0.2 and 75 ph cm$^{-2}$.\\

For each burst, we simulated 2000 light-curves that allowed us to generate 1000 cross-correlation functions. Following the procedure described above, we then determined the time delay by fitting the interval near the peak with an \emph{ad hoc} (e.g. Gaussian functions, combination of two Gaussian profiles having a common centroid and asymmetric double exponential functions). The top left (top right) panel in Figure~\ref{fig:sigma_sample} shows the distributions of the time delay uncertainties (each representing the standard deviation of 1000 Monte-Carlo simulations) estimated cross-correlating the sample of long (short) GRBs. We note that an accuracy equal or smaller than 1ms is obtained for 55\% of the long GRBs (Figure~\ref{fig:sigma_sample} top-left panel, red area), while an uncertainty equal or smaller than 5ms is obtained for 30\% of the short GRBs (Figure~\ref{fig:sigma_sample} top-right panel, red area). Finally, it should be emphasized that systematic cross-correlation simulations play a crucial role in terms of investigating systematic uncertainties on the method to localize the GRBs. The bottom left and right panels of Figure~\ref{fig:sigma_sample} represent the dependence of the cross-correlation accuracy as a function of the GRB flux for the long and short GRB, respectively. As expected, stronger GRBs allow to recover time delays with a better accuracy.    


\begin{figure}
\begin{center}
\begin{tabular}{cc}
\includegraphics[height=6cm]{Long_GRB_sigma_dist} &
\includegraphics[height=6cm]{short_GRB_sigma_dist} \\
\includegraphics[height=6cm]{Long_GRB_sigma_flux} &
\includegraphics[height=6cm]{short_GRB_sigma_flux} \\
\end{tabular}
\end{center}
\caption[example] 
%>>>> use \label inside caption to get Fig. number with \ref{}
{ \label{fig:sigma_sample} 
\emph{Top Left panel:} distribution of the delay accuracy estimated via cross-correlation techniques a random sample of 100 long GRB selected from the Fermi GBM catalogue. In red we highlighted the sub-sample (50\%)characterised by $\sigma_{cc} < 1$ ms. \emph{Top Right panel:} distribution of the delay accuracy estimated via cross-correlation techniques a random sample of 100 short GRB selected from the Fermi GBM catalogue. In red we highlighted the sub-sample (30\%) characterised by $\sigma_{cc} < 5$ ms. \emph{Bottom Left panel:} cross-correlation accuracy as a function of the GRB flux for the sample of long GRB. \emph{Bottom Right panel:} cross-correlation accuracy as a function of the GRB flux for the sample of short GRB.}
\end{figure}


\subsection{mission scenario}
Testing the localisation capabilities of the HTP/HSP mini-constellation requires the definition of a specific mission scenario. In the following, we will give a brief description of the experimental set up used to the analysis described in this work, that is the outcome of a long mission analysis investigation performed during the first year of the project.

\subsubsection{Low Earth Equatorial Orbit}
To achieve to scientific goals of the mission, the HTP/HSP orbit has been accurately investigated and finally restricted to a low Earth orbit with altitude between 500 and 600km and inclination <20deg. Within the specific framework of the analysis reported here, the adopted reference orbit have the following properties:

\begin{itemize}
\item altitude h=550 km;
\item circular orbit (eccentricity = 0);
\item Equatorial orbit (inclination = 0).
\end{itemize}


\subsubsection{Space segment injection strategy}
The space segments injection strategies explored in mission analysis are the following:
\begin{itemize}
\item Dedicated multiple injections - one per satellite - into different true anomalies at $t_0$; 
\item Single injection of each triplet with imposed relative motion among the spacecraft belonging to the same triplet, imposed by the deployer release spring.
\end{itemize} 

As resulted from the analysis, the first option is highly sensitive to the release conditions, e.g. natural perturbations provoke a relative drift which is emphasized by the launcher and deployer injection uncertainties jeopardizing this strategy robustness in terms of scientific outcome. This option requires a dedicated launch for the HERMES Pathfinder mission. 
On the other hand, the second option is feasible without a dedicated launch, since the triplet elements can be released in a single launch event with no dedicated injection manoeuvre. The relative motion between the satellites is actually imposed by the deployer spring authority which overcomes the natural perturbations effects and the launcher injection uncertainties, making more reliable the expected scientific outcomes foreseen in the design phase. 
Here, we will consider the second injection strategy to perform the localization test.

\begin{figure}[h!]
\centering
\includegraphics[scale=0.45,angle=-90]{pointying_strategy}
\vspace{-2.5cm}
\caption{LVLH optimal pointing strategy} 
\label{fig:strategy}
\end{figure}

\subsubsection{Pointing strategy}
\label{sec:po_strat}
The selection of the nominal pointing strategy significantly affects the HTP/HSP performances in terms of detection and localization of GRBs. During the mission analysis and the spacecraft design activity, specific trade-off studies have been carried out on the topic taking advantage of the quite advance attitude control performances of the HERMES CubeSats. In fact, the pointing direction of the payload’s Line Of Sight (LOS) can be controlled and even varied along the mission time-line, according to the short-medium planning for the payload utilization, compliant with the space segment capabilities, to maximize the science mission outcome. 
Three different pointing strategies have been explored:
\begin{itemize} 
\item Zenith pointing for each payload LOS; 
\item Co-alignment of $n\geq3$ payload LOSs on an Inertial-selected direction; 
\item Co-alignment of $n\geq3$ payload LOSs on a LVLH-selected direction (i.e. LOSs aligned on the zenith direction of a specific satellite in the fleet).
\end{itemize} 

The third pointing strategy, that will be used for the analysis discussed in this document, is preferred to maximize the scientific outcome of the mission. To do that, periodical optimization of the LOS direction of each satellite are foreseen in order to maximize the overlapping Field of View (FoV) and hence the number of GRBs potentially triangulated. More in details, the whole mission is divided in periods (from days to weeks) in which the pointing directions of each satellite is kept fixed in the non-inertial LVLH (Local Vertical-Local Horizontal) reference frame. The pointing direction of each satellite in the LVLH frame is uniquely defined by two angles: the first, in the orbital plane, is the angular displacement between the LOS and the radial direction, and the second is the elevation of the LOS above (or below) the orbital plane. During each period, the optimal set of angles (two for each satellites) is selected using a heuristic particle-swarm optimization algorithm to maximize the scientific return, i.e. the number of GRBs triangulated during that frame time. Figure~\ref{fig:strategy} shows the position and pointing of the HTP/HSP detectors within the LVLH pointing strategy (left panel) as well as the associated map of the overlapping FoVs.  



\subsubsection{Simulation strategy}
\label{sec:strat}
To investigate the level of accuracy on the detection and localization of GRBs with respect to the mission scenarios previously described we adopted the following strategy: 
\begin{itemize}
	
\item following the results reported in the Fourth Fermi-GBM GRB catalogue [\citenum{vonKienlin20}], we generated GRB events assuming a uniform distribution in the plane-of-the-sky. The number of simulated GRBs during the mission duration (2 years) reflects the Fermi GBM detection rate $\alpha_{GBM} \simeq 0.083$ GRB/sr/d, which corresponds approximately to a total of 760 GRB events in the whole sky ($4\pi$ steradians) within the assumed HTP/HSP mission life-time; 

\item we temporally located the 760 simulated GRB events by sampling uniformly at random from the time data-set associated to the positions of the fleet elements. More specifically, we randomly extracted 760 time-intervals (1-minute length) among the 1051201 available from the simulation of the satellite positions. With that, we are assuming that the process of detection of a GRB has a duration shorter or equal to 1 minute. We note that this assumption is surely valid for short GRBs, whilst it is not applicable for 30\%-40\% of the long GRBs showing duration larger than 60 seconds; 

\item for each GRB event, we determined the number of active satellites (non-transiting within the South Atlantic Anomaly) able to detect it, by verifying that the direction of the GRB and the LoS of the instrument identify an angle lower or equal 60 degrees. This guarantees that the GRB event falls within the $3\pi$ steradians FoV (Full Width at Half Maximum) of the detector. 

\item we processed only GRB events observed by a minimum of 3 satellites, or equivalently, for which a minimum of 3 satellites have the GRB in their FoVs. This guarantees the possibility to apply \emph{Temporal Triangulation} techniques to determine the position of the GRB in the sky. 

\end{itemize}

For each of the GRB potentially localizable, we determined the expected time delays $\Delta t_{ij}(\hat{d})$ between pairs of satellites by using Eq.~\ref{eq:delayexp}. The associated real (measured) time delay $\Delta \tau_{ij}$ will be then generated as:
\begin{equation} 
     \Delta \tau_{ij}=\Delta t_{ij}(\hat{d})+N(0,\sigma_{cc}),
\end{equation}              
where $N(0,\sigma_{cc})$ is the cross-correlation uncertainty randomly extracted assuming a normal distribution with zero mean and standard deviation equal to the results obtained from the analysis of the sample of long and short GRBs described above. More specifically, among the 760 GRBs, 83\% (following to Fermi GBM statistics) will be simulated assigning characteristics of the long GRBs and $\sigma_{cc-long} = 1$ ms, while the remaining 17\% will be simulated as short GRBs with $\sigma_{cc-short} = 5$ ms.
Using Eq.~\ref{eq:chisq}, we then calculated a $\chi^2$ map of the whole sky by creating a grid of points uniformly sampling the equatorial coordinates. Finally, we estimated the most probable position of the simulated GRB by calculating the values of $\alpha$ and $\delta$ that minimise the $\chi^2$ function. Confidence intervals at $1\sigma$ level represent the coordinate intervals $\sigma_\alpha$ and $\sigma_delta$ for which $\Delta \chi^2 (\hat{d}) = 
\Delta 
\chi^2(\nu, 68\%)$, where $\nu$ is the number of parameters in the function. We emphasise that the coordinate intervals obtained by marginalising the confidence region, especially $\sigma_\alpha$, can be overestimated. 


\begin{figure}
\begin{center}
\begin{tabular}{cc}
\includegraphics[scale=0.4,angle=0.0]{fig_loc_30deg} &
\includegraphics[scale=0.4,angle=90.0]{fig_loc_70deg}\\
a) & b)\\
\end{tabular}
\end{center}
\vspace{-3.5cm}
\caption[example] 
%>>>> use \label inside caption to get Fig. number with \ref{}
{ \label{fig:conf_reg} 
Example of GRB localization obtained by simulating a GRB with latitude < 30 deg (left panel) and latitude > 70 deg (right panel). The GRB position and best-fit position are represented with the green X-symbol and the red +-symbol, respectively.  The red contour line shows the 1$\sigma$ confidence level region associated with the best fit position. The blue-filled region represents the overlapping FoVs of the 3 satellites observing simultaneously the GRB. The light-green contour line shows the 1$\sigma$ confidence level region taking into account the FoVs of the detectors.}
\end{figure}



In Figure~\ref{fig:conf_reg} we report an example of two possible outcomes obtained from the simulations described above. Panels a) and b) the figure summarise the results of the simulation displaying the GRB real position in the plane of the sky (green X-symbol), the best-fit position of the GRB (red +symbol) and the corresponding $1\sigma$ confidence region shown both in spherical and bi-dimensional representations. Moreover, in light-green we report the positional $1\sigma$ confidence region corrected by overlapping FoVs of detectors (blue-filled region) observing simultaneously the event. Both cases describe the localisation of GRB events detected simultaneously by 3 satellites. While for the event shown in the panel a) of Figure~\ref{fig:conf_reg} the confidence region of the GRB coordinates is nearly unconstrained in the $\delta$ coordinate, panel b) describes a more favourable scenario in which the uncertainty on the GRB position is limited to a relatively small region. For the former case, we note that, although limited, two confidence regions are present but only one includes the position of the GRB. This ambiguity reflects the planar symmetry of the signal delays with respect to the satellite orbital plane. Differences between these two cases are the superposition of several aspects such as the accuracy in the determination of the time signal time delays, physical distances between the satellites and projected distances of the satellites with respect to the direction of the GRB event. Moreover, for the two cases discussed above (but also for the larger set of results obtained with the simulations) we note that the Right Ascension is always relatively well constrained, leaving the largest uncertainty in the Declination. This result is expected when considering triplets of satellites laying in the equatorial plane. Improvements in $\delta$ can be obtained by increasing the number of satellites simultaneously observing the GRB event and located in inclined plane with respect to the equatorial one.



\section{Results and Discussion}

We applied the method described in Sec.~\ref{sec:strat} for each of the 760 GRBs simulated within the 2-years time interval characterising the life time of the HTP/HSP mission. To estimate confidence intervals of these quantities, we performed 1000 Monte-Carlo simulations in which we applied the method extensively discussed earlier on. We note that by assuming the $\sigma_{cc-long}$ and $\sigma_{cc-short}$ values reported in Sec.~\ref{sec:grbcc} for all the simulated long and short GRBs, respectively, we are not correctly representing the GRB sample propertied. To mitigate this issue, the results reported below have been rescaled to take into account that $\sigma_{cc-long}$ and $\sigma_{cc-short}$ characterise only 55\% and 30\% of the long and the short GRB populations investigated, respectively.



\subsection{Long GRBs}

Figure~\ref{fig:res_long} shows an example of the distribution of $\sigma_\alpha$ and $\sigma_\delta$ obtained from the observed GRBs within a 2-year long simulation. We emphasise that the number of observed GRBs reported in Figure~\ref{fig:res_long} needs to be rescaled by a factor of $\sim1.8$ to account for the assumed value of delay accuracy achievable $\sigma_{cc-long}$. In red and white (hatched region) we highlighted the detected GRBs for which the uncertainties on the two coordinates are lower or equal to 30 degrees (40 degrees for $\delta$) and 15 degrees, respectively. Figure~\ref{fig:res_long} clearly confirms the expected capabilities of the HTP/HSP set up, that allows us to be more sensitive to the Right Ascension with respect to the Declination coordinate.  
\vspace{-3cm}
\begin{figure}[h!]
\centering
\includegraphics[scale=0.6,angle=0]{res_long_pos}
\vspace{-4cm}
\caption{Distribution of the Right Ascension (left panel) and Declination (right panel) uncertainties from one of simulations of long GRBs. The red and white-hatched regions highlight the GRBs with $\alpha$ and $\delta$ uncertainties lower than 30 degrees (40 degrees for $\delta$) and 15 degrees, respectively.} 
\label{fig:res_long}
\end{figure}

Combining the results from the 1000 Monte-Carlo simulations, we obtained that among the nearly 630 long GRBs generated in each simulation, $240\pm11$ are detected simultaneously by at least 3 detectors. We can further identify the following quantities:
\begin{itemize}
\item $66\pm5$ events with $\sigma_\alpha < 30$ deg;
\item	$20\pm3$ events with $\sigma_\alpha < 30$ deg and $\sigma_\delta < 40$ deg;
\item	$57\pm4$ events with $\sigma_\alpha < 15$ deg;
\item	$16\pm2$ events with $\sigma_\alpha < 15$ deg and $\sigma_\delta < 40$ deg;
\item	$7\pm2$ events with $\sigma_\alpha < 5$ deg;
\item	$4\pm1$ events with $\sigma_\alpha < 5$ deg and $\sigma_\delta < 40$ deg;
\item	$1\pm0.6$ events with $\sigma_\alpha < 5$ deg and $\sigma_\delta < 10$ deg;
\end{itemize}


\subsection{Short GRBs}

Figure~\ref{fig:res_short} shows the distribution of $\sigma_\alpha$ and $\sigma_\delta$ obtained from one simulation of the observed GRBs within a 2-year long mission (to be rescaled by a factor of $\sim3$ to consider the assumed value of $\sigma_{cc-short}$). Red and white(hatched) regions represent the detected GRBs for which the uncertainties on the $\alpha$ coordinate are lower or equal to 30 degrees and 15 degrees, respectively. Combining the results from the 1000 Monte-Carlo simulations, we obtained that among the $\sim130$ short GRBs generated in each simulation, $48\pm5$ are detected simultaneously by at least 3 detectors. As clear from Figure~\ref{fig:res_short}, short GRBs are less likely to be precisely localised with the adopted pointing strategy. In fact, the combination of the relatively large cross-correlation uncertainty ($\sigma_{cc-short} = 5$ ms) and the projected baselines allows us to predict only $2\pm1$ short GRBs located with $\sigma_\alpha < 30$ deg. In the following section we will investigate alternative approaches on the pointing strategy to improve the localisation capabilities for long and short GRBs.

\vspace{-3cm}
\begin{figure}[h!]
\centering
\includegraphics[scale=0.6,angle=0]{res_short_pos}
\vspace{-4cm}
\caption{Distribution of the Right Ascension (left panel) and Declination (right panel) uncertainties from one of simulations of short GRBs. The red and white-hatched regions highlight the GRBs with $\alpha$ uncertainties lower than 30 degrees and 15 degrees, respectively.} 
\label{fig:res_short}
\end{figure}


\subsection{Localisation capabilities v.s. satellite baselines}
As discussed in Sec.\ref{sec:po_strat}, it is worth stressing that the pointing strategy LVLH aims at optimising the overlapping FoVs of the detectors composing a triplet of satellites, in order to guarantee the maximisation of the number of GRBs observed with the possibility to potentially localise them by means of \emph{temporal triangulation}. As clear from the analysis reported above, a large number of events observed simultaneously by 3 (or more) satellites, although desirable, does not guarantee a large number of GRBs well located in the sky. With that in mind, adjustments on the pointing strategy could be applied to improve the localisation capabilities of the HTP/HSP mini-constellation.
Combining Eq. 2 and Eq.~\ref{eq:chisquare}, it is possible to deduce that the key elements to accurately locate an event in the sky are the accuracy in determining the time delay between the observations in different detectors as well as their projected distance with respect the observed event. In Sec. 4.4 we extensively discussed the former aspect, while here we will focus on the latter by discussing different aspects of the performed simulations. We start by exploring possible correlations between the projected baseline, parameter that describes the relative positions of two satellites with respect a generic GRB direction in the sky, and the accuracy on the two coordinates RA and DEC used to localise an event in the sky. Taking as a reference a triplet of satellite, we can define three different projected baselines, one of which will depend upon the others. As a reference parameter we consider the smallest of the projected baselines defined within the triplet, we then take the whole dataset of long and short simulated GRBs for a specific orbit and we correlate them against the minimum projected baselines. 

\begin{figure}[h!]
\centering
\includegraphics[scale=0.6,angle=0]{fig_all_pos_long_sim}
\vspace{-4cm}
\caption{Uncertainties on the Right Ascension $\sigma_\alpha$ (left panel) and Declination $\sigma_\alpha$ (right panel) coordinates as a function of the minimum projected baseline within the triplet (or multiplet) of satellites that observed simultaneously the simulated long GRBs.} 
\label{fig:all_pos_long_sim}
\end{figure}


Figure~\ref{fig:all_pos_long_sim} shows an example of the correlation between $\sigma_\alpha$ (left panel) and $\sigma_\delta$ (right panel) with respect to the minimum projected baseline for all the long GRBs. The two plots clearly show a dispersion of the values $\sigma_\alpha$ and $\sigma_\delta$. Interestingly and in line with the predictions, this dispersion as well as the absolute value of the uncertainty significantly decrease at large values of the minimum projected baseline. To further investigate the correlations shown in Figure~\ref{fig:all_pos_long_sim}, we apply binning techniques to the datasets creating equally spaced intervals in the projected baseline parameter and averaging the values of $\sigma_\alpha$ and $\sigma_\delta$ within the selections. Figure~\ref{fig:all_pos_long_reb} (empty circles) shows the results of the binning method for the long GRB dataset of Figure~\ref{fig:all_pos_long_sim}. The figure describes the average localisation capabilities of the HTP/HSP configuration while observing randomly distributed long GRBs characterised by $\sigma_{cc-long} = 1$ ms. It is interesting to note that both the uncertenties on the equatorial coordinates decrease coherently with increasing values of the minimum projected baseline. To be able to quantify the variation of these parameters, we modelled the data with a generic power-law function $a\cdot x^b$. Based on the best-fit models (solid black lines in Figure~\ref{fig:all_pos_long_sim}), we can extrapolate the localisation capabilities of the mini-constellation at different values of the minimum projected baseline of a generic triplet defined within the LVLH pointing strategy. It is worth to notice that by operating the two triplets with an average minimum projected baseline of the order of 6000 km, the mini-constellation will be able to routinely locate all the long GRBs characterised by $\sigma_{cc-long} = 1$ ms (half of observable sample) with an accuracy $\sigma_\alpha \leq 10$ deg and $\sigma_\delta \leq 42$ deg (red-dashed lines). By increasing the minimum projected baseline up to 10000 km, the accuracies will become of the order of $\sigma_\alpha \leq 7$ deg and $\sigma_\alpha \leq 35$ deg (cyan-dashed lines). 

\begin{figure}[h!]
\centering
\includegraphics[scale=0.6,angle=0]{fig_all_pos_long_reb}
\vspace{-4cm}
\caption{Average correlation between the uncertainties in the Right Ascension $\sigma_\alpha$ (left panel) and Declination $\sigma_\alpha$ (right panel) coordinates as a function of the minimum projected baseline obtained by binning the results from the simulated long GRB as detected by the HTP/HSP mini-constellation. The black-solid lines represent the best-fitting model to the data, whilst dashed red and cyan lines mark the achievable values of $\sigma_\alpha$ and $\sigma_\alpha$ for minimum projected baselines of 6000 km and 10000 km, respectively} 
\label{fig:all_pos_long_reb}
\end{figure}

We performed a similar analysis for the short GRBs. In analogy with the long ones, Figure~\ref{fig:all_pos_short_reb} (empty circles) represents the average localisation capabilities of the HTP/HSP configuration with respect to randomly distributed short GRBs characterised by $\sigma_{cc-long}s = 5$ ms. Also in this case, both the uncertainties on the equatorial coordinates decrease coherently with increasing values of the minimum projected baseline, although the behaviour at large value of the baseline starts to be more complex. To be able to quantify the variation of these parameters, we modelled the data with an \emph{ad hoc} function reported in the figure legend. Based on the best-fit models (solid black lines in figure), we can extrapolate the localisation capabilities of the mini-constellation at different values of the minimum projected baseline of a generic triplet. It is interesting to notice that by operating the two triplets with an average minimum projected baseline of the order of 6000 km it will be possible to routinely locate the short GRBs characterised by $\sigma_{cc-short} = 5$ ms (~30\% of the complete sample) with an accuracy $\sigma_\alpha \leq 50$ deg and $\sigma_\delta \leq 85$ deg (red-dashed lines). Increasing the minimum projected baseline up to 10000 km will allow to reduce the accuracies at values $\sigma_\alpha \leq 28$ deg and $\sigma_\delta \leq 35$ deg (cyan-dashed lines).

It is noteworthy that, from simple geometrical considerations, increasing the projected baseline of the satellites will imply the reduction of overlapping FoV of the detectors with a consequent decrease in number of GRBs detected. On the other hand, the fraction of events with good localisation will significantly increase. Therefore, a trade-off study will be carried out to define adjustments on the mission strategy able to maximise the scientific outcomes by defining the compromising number of accurately localised GRBs events.

\begin{figure}[h!]
\centering
\includegraphics[scale=0.6,angle=0]{fig_all_pos_short_reb}
\vspace{-4cm}
\caption{Average correlation between the uncertainties in the Right Ascension $\sigma_\alpha$ (left panel) and Declination $\sigma_\alpha$ (right panel) coordinates as a function of the minimum projected baseline obtained by binning the results from the simulated short GRB as detected by the HTP/HSP mini-constellation. The black-solid lines represent the best-fitting model to the data, whilst dashed red and cyan lines mark the achievable values of $\sigma_\alpha$ and $\sigma_\alpha$ for minimum projected baselines of 6000 km and 10000 km, respectively} 
\label{fig:all_pos_short_reb}
\end{figure}

\section{Summary}

In this paper, we characterised the localisation capabilities of the HTP/HSP mini-constellation based on the application of \emph{Temporal Triangulation} methods on the observed GRBs. We extensively investigated the validity of cross-correlation techniques to accurately recover time delays between the GRB light-curves simultaneously detected by elements of constellation. By studying unbiased samples of 100 long and short GRBs, we deeply studied correlations between GRB properties and accuracy on measuring delays by cross-correlating their light-curves. Finally, based on specific mission scenario and an optimised pointing strategy for the detectors, we simulated the HTP/HSP performances during a 2-years lifetime mission by providing an estimate of the number of localised GRBs. Moreover, we provide estimates of the positional accuracies of the detected GRBs depending on their properties. Finally, we investigated variations of the observational set-up to improve the localisation capabilities of the mini-constellation. Based on the results obtained from these simulations, we can conclude that the HERMES Pathfinder will achieve its scientific goal of collected enough GRB detections and localizations to validate the HERMES overall concept, as well as proving crucial insights for the design of an extended version of the constellation.


%As a final remark, we emphasise that these results and predictions are part of an on-going investigation of the HTP/HSP scientific performances, therefore updates as well as improvements in the techniques applied to the data analysis are be expected. Indeed, we are currently investigating the localisation capabilities of the HTP-HSP constellation combining the technique previously described in combination with an independent approach that takes advantage of the differences in measured count rates from the detectors observing the source at different aspect angles (similar to the technique used by Fermi GBM to localise the observed GRBs). With this strategy, we expect to be able to reduce the positional uncertainty, however, further investigation are required to be able to quantify the improvement.




\acknowledgments % equivalent to \section*{ACKNOWLEDGMENTS}       
 
This work has been carried out in the framework of the HERMES-TP and HERMES-SP collaborations. We acknowledge support from the European Union Horizon 2020 Research and Innovation Framework Programme under grant agreement HERMES-Scientific Pathfinder n. 821896 and from ASI-INAF Accordo Attuativo HERMES Technologic Pathfinder n. 2018-10-H.1-2020.  

% References
\bibliography{report} % bibliography data in report.bib
\bibliographystyle{spiebib} % makes bibtex use spiebib.bst

\end{document} 
